\documentclass[12pt,a4paper]{article}
\title{Harnessing Wasted Computing Power for Scientific Computing} 
\author{S\'andor Guba, M\'at\'e \H{O}ry and Imre Szeber\'enyi\\
Budapest University of Technology and Economics, %\\
%Magyar Tud\'osok k\"or\'utja 2, H-1117 Budapest,
Hungary}
\date{\empty}

\begin{document} 
\maketitle


Nowadays more and more general purpose workstations installed in a student
laboratory have built in multi-core CPU and graphics card providing significant
computing power. In most cases the utilization of these resources is low, and
limited to lecture hours. The concept of utility computing plays an important
role in nowadays technological development. As part of utility computing, cloud
computing offers greater flexibility and responsiveness to ICT users at lower
cost.

In  this paper, we introduce a cloud management system which enables the
simultaneous use of both dedicated resources and opportunistic environment. All
the free workstations (powered or not) are automatically added to a resource
pool, and can be used like ordinary cloud resources. Researchers can launch
various virtualized software appliances. Our solution leverages the advantages
of HTCondor and OpenNebula systems.

Modern graphics processing units (GPUs) with many-core architectures have
emerged as general-purpose parallel computing platforms that can dramatically
accelerate  scientific applications used for various simulations. Our business
model harnesses computing power of GPUs as well, using the needed amount of
unused machines. This makes the infrastructure flexible and power efficient.

Our pilot infrastructure consist of a high performance cluster and 28
workstations with dual-core CPUs and dedicated graphics cards. Altogether we
can use 10,752 CUDA cores through the network.

\end{document}
